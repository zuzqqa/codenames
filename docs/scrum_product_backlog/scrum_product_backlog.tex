%%%%%%%%%%%%%%%%%%%%%%%%%%%%%%%%%%%%%%%%%%%%%%%%%%%%%%%%%%%%%%%%
% DOCUMENT DEFINITION AND BASIC SETTINGS
%%%%%%%%%%%%%%%%%%%%%%%%%%%%%%%%%%%%%%%%%%%%%%%%%%%%%%%%%%%%%%%%
\documentclass[12pt,a4paper,colorlinks=true,linkcolor=NavyBlue,citecolor=red,urlcolor=NavyBlue]{book}
\usepackage[utf8]{inputenc}
\usepackage[T1]{fontenc}
\usepackage[polish]{babel}         

%%%%%%%%%%%%%%%%%%%%%%%%%%%%%%%%%%%%%%%%%%%%%%%%%%%%%%%%%%%%%%%%
% FONT AND STYLE PACKS
%%%%%%%%%%%%%%%%%%%%%%%%%%%%%%%%%%%%%%%%%%%%%%%%%%%%%%%%%%%%%%%%
\usepackage{mathpazo}             
\usepackage[semibold]{sourcesanspro}
\usepackage{sectsty}
\allsectionsfont{\sffamily} 

%%%%%%%%%%%%%%%%%%%%%%%%%%%%%%%%%%%%%%%%%%%%%%%%%%%%%%%%%%%%%%%%
% COLORS AND GRAPHIC STYLES
%%%%%%%%%%%%%%%%%%%%%%%%%%%%%%%%%%%%%%%%%%%%%%%%%%%%%%%%%%%%%%%%
\usepackage{color}
\definecolor{Valentia}{RGB}{233,78,82}
\definecolor{Titleblue}{RGB}{114,146,162}

%%%%%%%%%%%%%%%%%%%%%%%%%%%%%%%%%%%%%%%%%%%%%%%%%%%%%%%%%%%%%%%%
% LAYOUT AND FORMATTING PACKAGES
%%%%%%%%%%%%%%%%%%%%%%%%%%%%%%%%%%%%%%%%%%%%%%%%%%%%%%%%%%%%%%%%
\usepackage{mdframed}
\usepackage{multirow}
\usepackage{multicol}
\usepackage{tikz}
\usepackage{graphicx}
\usepackage[absolute]{textpos}
\usepackage{colortbl}
\usepackage{array}
\usepackage{geometry}
\usepackage{scrextend}

%%%%%%%%%%%%%%%%%%%%%%%%%%%%%%%%%%%%%%%%%%%%%%%%%%%%%%%%%%%%%%%%
% HEADER AND FOOTER SETTINGS
%%%%%%%%%%%%%%%%%%%%%%%%%%%%%%%%%%%%%%%%%%%%%%%%%%%%%%%%%%%%%%%%
\usepackage{fancyhdr}
\pagestyle{fancy}
\fancyhf{}
\fancyheadoffset{0.005\textwidth}
\fancyhead[LE]{\thepage}
\fancyhead[RE]{\nouppercase\leftmark}
\fancyhead[RO]{\thepage}
\fancyhead[LO]{\nouppercase\rightmark}

%%%%%%%%%%%%%%%%%%%%%%%%%%%%%%%%%%%%%%%%%%%%%%%%%%%%%%%%%%%%%%%%
% HYPERLINKS AND CHAPTER STYLES
%%%%%%%%%%%%%%%%%%%%%%%%%%%%%%%%%%%%%%%%%%%%%%%%%%%%%%%%%%%%%%%%
\usepackage[Sonny]{fncychap}
\usepackage[colorlinks=true, linkcolor=black, citecolor=black, urlcolor=black]{hyperref}

%%%%%%%%%%%%%%%%%%%%%%%%%%%%%%%%%%%%%%%%%%%%%%%%%%%%%%%%%%%%%%%%
% BEGINNING OF THE DOCUMENT
%%%%%%%%%%%%%%%%%%%%%%%%%%%%%%%%%%%%%%%%%%%%%%%%%%%%%%%%%%%%%%%%
\begin{document}

%%%%%%%%%%%%%%%%%%%%%%%%%%%%%%%%%%%%%%%%%%%%%%%%%%%%%%%%%%%%%%%%
% TITLE PAGE
%%%%%%%%%%%%%%%%%%%%%%%%%%%%%%%%%%%%%%%%%%%%%%%%%%%%%%%%%%%%%%%%
\begin{titlepage}
\newgeometry{left=2.5cm, bottom=3cm, top=2cm, right=2.5cm}

\tikz[remember picture,overlay] \node[opacity=0.03,inner sep=0pt] at (73.6mm, -124.25mm){\includegraphics{characters.png}};

\centering
\color{black}
\fontsize{24}{13}\selectfont
\textbf{DOKUMENT PROJEKTU} \\[2mm]
\normalsize
\color{black}
\bigskip
\textbf{Wersja dokumentu: 1.1}\\[1mm]
\bigskip
\textbf{Data utworzenia: 2.04.2025}\\[1mm]
\bigskip
\textbf{Data ostatniej aktualizacji: 2.04.2025}

% Title of the project
\color{black}
\vspace{2cm}
{\fontsize{28}{32} \selectfont \textbf{Gra internetowa}}\\ 
\vspace{0.3cm} 
{\fontsize{45}{32} \selectfont \textbf{Codenames}} 

% Subtitle of the project
\vspace{2cm}
\fontsize{15}{18}\selectfont
\color{black}
\textbf{Scrum: Backlog produktu\\}
\bigskip
\vspace{5cm}

% Information about authors
\normalsize
\bigskip
\fontsize{12}{12}\selectfont
\vspace{1.5mm}
\raggedright
\begin{tabular}{ll}
\textbf{Redaktor:} & Agata Domasik \\[6mm]
\textbf{Współautorzy:} & Zuzanna Nowak \\[2mm]
& Adam Chabraszewski \\[2mm]
& Jakub Walasik \\[6mm]
\textbf{Liczba stron:} & 12 \\[2mm]
\end{tabular}

% Logo
\vspace{\fill}
\begin{center}
    \includegraphics[scale=0.3]{logo.png} 
\end{center}
\vspace{-15mm}
\end{titlepage}

%%%%%%%%%%%%%%%%%%%%%%%%%%%%%%%%%%%%%%%%%%%%%%%%%%%%%%%%%%%%%%%%
% GEOMETRY SETTINGS FOR THE MAIN BODY OF THE DOCUMENT
%%%%%%%%%%%%%%%%%%%%%%%%%%%%%%%%%%%%%%%%%%%%%%%%%%%%%%%%%%%%%%%%
\newgeometry{top=2cm, bottom=2.5cm, left=2cm, right=2cm}

%%%%%%%%%%%%%%%%%%%%%%%%%%%%%%%%%%%%%%%%%%%%%%%%%%%%%%%%%%%%%%%%
% CONTENT OF THE DOCUMENT
%%%%%%%%%%%%%%%%%%%%%%%%%%%%%%%%%%%%%%%%%%%%%%%%%%%%%%%%%%%%%%%%

%%%%%%%%%%%%%%%%%%%%%%%%%%%%%%%%%%%%%%%%%%%%%%%%%%%%%%%%%%%%%%%%
% TABLE OF CONTENTS
%%%%%%%%%%%%%%%%%%%%%%%%%%%%%%%%%%%%%%%%%%%%%%%%%%%%%%%%%%%%%%%%
\tableofcontents

\chapter{Wprowadzenie - o dokumencie}
\section{Cel dokumentu}
Celem zadania jest opisanie produktu wytwarzanego w ramach projektu. Produkt
przybliżany jest poprzez biznesowy scenariusz jego użycia,
z którego następnie wywodzone są cechy produktu
dokumentowane w backlogu produktu z priorytetami. 
\section{Odbiorcy}

\begin{itemize}
    \item Dr inż. Jakub Miler - prowadzący przedmiot \textit{Realizacja projektu informatycznego},
    \item Dr inż. Katarzyna Łukasiewicz - prowadzący zajęcia projektowe,
    \item Katedra Inżynierii Oprogramowania, \\[2mm] 
Wydział Elektroniki, Telekomunikacji i Informatyki, \\[2mm]  
Politechnika Gdańska,
    \item Członkowie zespołu projektowego:
    \begin{itemize}
        \item[] Zuzanna Nowak, 193165 - kierownik projektu
        \item[] Agata Domasik, 193577
        \item[] Jakub Walasik, s193650
        \item[] Adam Chabraszewski, s193373
    \end{itemize}
\end{itemize}




%%%%%%%%%%%%%%%%%%%%%%%%%%%%%%%%%%%%%%%%%%%%%%%%%%%%%%%%%%%%%%%%
% Chapters
%%%%%%%%%%%%%%%%%%%%%%%%%%%%%%%%%%%%%%%%%%%%%%%%%%%%%%%%%%%%%%%%
\chapter{Backlog produktu}
\section{O projekcie i produkcie}
Gra Codenames to cyfrowa adaptacja popularnej gry towarzyskiej, w której gracze podzieleni na dwie drużyny próbują odgadnąć słowa na planszy na podstawie podpowiedzi od swojego kapitana. Wersja komputerowa umożliwia rozgrywkę online z innymi graczami oraz wprowadza nowe funkcje, takie jak lobby i komunikacja głosowa.
\vspace{1cm}

\section{Persony użytkowników}
\begin{enumerate}
    \item Kasia, 28 lat, fanka gier planszowych
    \begin{itemize}
        \item[•] Wiek: 28 lat
        \item[•] Zawód: Pracuje jako grafik w agencji reklamowej w Krakowie
        \item[•]Styl życia: Towarzyska, lubi spotkania ze znajomymi, szczególnie przy planszówkach i winie.
        \item Problemy:
        \begin{itemize}
            \item[•] Coraz trudniej zorganizować spotkania na żywo ze względu na pracę i obowiązki znajomych.
            \item[•] Brakuje jej wygodnej platformy do grania w planszówki online z bliskimi.
            \item[•] Frustrują ją skomplikowane interfejsy aplikacji
        \end{itemize}
        \item Potrzeby i oczekiwania:
        \begin{itemize}
            \item[•] Intuicyjna, estetyczna i łatwa w obsłudze platforma.
            \item[•] Możliwość tworzenia prywatnych gier z hasłem, tylko dla zaproszonych znajomych.
            \item[•] Wbudowany czat głosowy, by można było swobodnie rozmawiać podczas gry – jak przy stole.
        \end{itemize}
    \end{itemize}
    \item Piotr
    \begin{itemize}
        \item[•] Wiek: 19 lat
        \item[•] Zawód: Student informatyki na Politechnice Wrocławskiej
        \item[•]Styl życia: Spędza dużo czasu przy komputerze – zarówno ucząc się, jak i grając. Lubi techniczne nowinki, streamuje gry od czasu do czasu.
        \item Problemy:
        \begin{itemize}
            \item[•] Trudno mu znaleźć godnych przeciwników w planszówki online.
            \item[•] Brakuje mu motywacji do grania, gdy nie ma systemu postępu.
        \end{itemize}
        \item Potrzeby i oczekiwania:
        \begin{itemize}
            \item[•] Szybkie i sprawne publiczne lobby, gdzie łatwo znajdzie grę bez czekania.
            \item[•] System rankingowy i statystyki, by móc śledzić progres i porównywać się z innymi.
            \item[•] Wbudowany czat tekstowy – przydatny do komunikacji z innymi graczami, zwłaszcza przy grach drużynowych lub z losowymi przeciwnikami.
        \end{itemize}
    \end{itemize}
\end{enumerate}



\section{Scenariusz użycia produktu}
\begin{enumerate}
    \item \textbf{Kasia organizuje grę ze znajomymi w prywatnym lobby}  \\[2mm] 
    Kasia chce spędzić wieczór z przyjaciółmi, ale nie mogą się spotkać osobiście. Postanawia zorganizować rozgrywkę online w Codenames.
    \begin{itemize}
        \item[•] Loguje się do gry, wybierając logowanie przez e-mail.
        \item[•] Tworzy prywatne lobby, nadając mu nazwę i ustawiając hasło dostępu.
        \item[•] Znajomi wyszukują lobby i dołączają do gry, wpisując hasło.
        \item[•] Wszyscy uczestnicy włączają czat głosowy, aby móc swobodnie się komunikować.
        \item[•] Drużyny decydują, kto zostanie kapitanem.
        \item[•] Gra rozpoczyna się, a drużyny na zmianę próbują odgadnąć hasła na planszy.
        \item[•] Po zakończonej rozgrywce pokazują statystyki – ilość słów odgadnętych przez drużyny.
    \end{itemize}
    \item \textbf{Piotr dołącza do publicznego lobby i rywalizuje z nieznajomymi}   \\[2mm] 
    Piotr ma wolny wieczór i chce zagrać w Codenames, ale nie ma grupy do wspólnej gry. Decyduje się dołączyć do publicznego lobby.
    \begin{itemize}
        \item[•] Loguje się do gry, korzystając z konta Google.
        \item[•] Wybiera publiczne lobby, do którego chce dołączyć
        \item[•] System automatycznie przypisuje go do drużyny 
        \item[•] Gra rozpoczyna się, a Piotr zostaje kapitanem swojej drużyny.
        \item[•] Wpisuje podpowiedź składającą się z jednego słowa i liczby, a jego drużyna próbuje odgadnąć słowa.
        \item[•] W trakcie rozgrywki Piotr korzysta z czatu tekstowego, aby komunikować się z drużyną.
        \item[•] Po zakończonej grze otrzymuje punkty rankingowe i sprawdza swoje miejsce w tabeli liderów.
    \end{itemize}
\end{enumerate}

\vspace{1cm}
\section{Backlog Produktu}
\begin{figure}[h!]
    \centering
    \vspace{0.5cm}
    \includegraphics[width=1\textwidth]{Backlog.png} 
    \caption{Lista elementów backlogu}
    \label{fig:backlog}
\end{figure}

\vspace{1cm}
{\large Lista elementów backlogu jest posortowana według priorytetów:}
\begin{itemize}
    \item \textbf{Bardzo wysoki} – kluczowe dla podstawowego działania gry.
    \item \textbf{Wysoki} - ważne, ale nie blokujące działania gry.
    \item \textbf{Średni} – dość ważne dla pewnych grup użytkowników
    \item \textbf{Niski} – funkcjonalności dodatkowe, możliwe do dodania później.
\end{itemize}



\newpage
\section{Kryteria akceptacji}

\begin{figure}[h!]
    \centering
    \includegraphics[width=1\textwidth]{kryteria3.png} 
    \caption{Kryteria akceptacji dla funkcjonalności \textit{wysyłanie podpowiedzi}}
\end{figure}

\begin{figure}[h!]
    \centering
    \vspace{1cm}
    \includegraphics[width=1\textwidth]{kryteria1.png}
    \caption{Kryteria akceptacji dla funkcjonalności \textit{lobby prywatne}}
    \vspace{1cm}
\end{figure}

\begin{figure}[h!]
    \centering
    \vspace{1cm}
    \includegraphics[width=1\textwidth]{kryteria2.png}
    \caption{Kryteria akceptacji dla funkcjonalności \textit{czat tekstowy}}
    \vspace{1cm}
\end{figure}

\begin{figure}[h!]
    \centering
    \vspace{1cm}
    \includegraphics[width=1\textwidth]{kryteria4.png}
    \caption{Kryteria akceptacji dla funkcjonalności \textit{autentykacja Google}}
    \vspace{0.5cm}
\end{figure}


\vspace{1cm}
{\large Ogólne kryteria akceptacyjne:}
\begin{itemize}
\item \textbf Stabilność i działanie podstawowych funkcji - gra działa bezawaryjnie, wszystkie podstawowe funkcje gry działają zgodnie z założeniami

\item \textbf{Interfejs użytkownika (UI/UX)} - Interfejs jest zgodny z projektem i intuicyjny.
Wszystkie komunikaty, etykiety i opcje są w pełni przetłumaczone i poprawne językowo.

\item \textbf{Bezpieczeństwo} - wszystkie dane użytkowników (np. loginy, hasła, dane osobowe) są odpowiednio zabezpieczone, zaimplementowane są odpowiednie środki ochrony przed atakami typu SQL injection, XSS, CSRF itp.

\item \textbf{Wydajność} - gra działa płynnie, nawet w przypadku dużej liczby graczy, czas ładowania gry jest akceptowalny (poniżej 3 sekund).

\item \textbf{Testy} - produkt przeszedł wszystkie testy (testy jednostkowe, integracyjne, funkcjonalne, UX), testy manualne zostały przeprowadzone z udziałem użytkowników docelowych

\item \textbf{Zgodność z wymaganiami biznesowymi} - wszystkie user stories zostały zrealizowane i zaakceptowane przez interesariuszy.

\item \textbf{Dokumentacja} - cała dokumentacja (techniczna, użytkowa) jest dostępna i aktualna.
\end{itemize}
\vspace{0.5cm}

\newpage
\section{Definicja ukończenia}

\begin{enumerate}
    \item Wykonanie wszystkich wymaganych zadań - wszystkie funkcje opisane w user story zostały zaimplementowane zgodnie z wymaganiami
    \item Przeprowadzenie testów - testy jednostkowe, integracyjne, funkcjonalne, UI/UX, wydajnościowe
    \item Przegląd kodu - Kod przeszedł proces przeglądu kodu przez co najmniej jednego członka zespołu (code review).
    \item Zatwierdzenie przez interesariuszy: zatwierdzono, że funkcjonalność działa zgodnie z wymaganiami.
    \item Utworzenie dokumentacji
    \item Gotowość do wdrożenia
\end{enumerate}


\newpage

%%%%%%%%%%%%%%%%%%%%%%%%%%%%%%%%%%%%%%%%%%%%%%%%%%%%%%%%%%%%%%%%
% END PAGE
%%%%%%%%%%%%%%%%%%%%%%%%%%%%%%%%%%%%%%%%%%%%%%%%%%%%%%%%%%%%%%%%
\newpage
\thispagestyle{empty}
\tikz[remember picture,overlay] \node[opacity=0.03,inner sep=0pt] at (73.6mm, -124.25mm){\includegraphics{characters.png}};
\begin{center}
    \vspace*{\fill}
    \includegraphics[width=0.3\textwidth]{logo.png} 
    \vspace*{\fill}
\end{center}

%%%%%%%%%%%%%%%%%%%%%%%%%%%%%%%%%%%%%%%%%%%%%%%%%%%%%%%%%%%%%%%%
% END OF DOCUMENT
%%%%%%%%%%%%%%%%%%%%%%%%%%%%%%%%%%%%%%%%%%%%%%%%%%%%%%%%%%%%%%%%
\end{document}