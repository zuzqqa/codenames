%%%%%%%%%%%%%%%%%%%%%%%%%%%%%%%%%%%%%%%%%%%%%%%%%%%%%%%%%%%%%%%%
% DOCUMENT DEFINITION AND BASIC SETTINGS
%%%%%%%%%%%%%%%%%%%%%%%%%%%%%%%%%%%%%%%%%%%%%%%%%%%%%%%%%%%%%%%%
\documentclass[12pt,a4paper,colorlinks=true,linkcolor=NavyBlue,citecolor=red,urlcolor=NavyBlue]{book}
\usepackage[utf8]{inputenc}
\usepackage[T1]{fontenc}
\usepackage[polish]{babel}         

%%%%%%%%%%%%%%%%%%%%%%%%%%%%%%%%%%%%%%%%%%%%%%%%%%%%%%%%%%%%%%%%
% FONT AND STYLE PACKS
%%%%%%%%%%%%%%%%%%%%%%%%%%%%%%%%%%%%%%%%%%%%%%%%%%%%%%%%%%%%%%%%
\usepackage{mathpazo}             
\usepackage[semibold]{sourcesanspro}
\usepackage{sectsty}
\allsectionsfont{\sffamily} 

%%%%%%%%%%%%%%%%%%%%%%%%%%%%%%%%%%%%%%%%%%%%%%%%%%%%%%%%%%%%%%%%
% COLORS AND GRAPHIC STYLES
%%%%%%%%%%%%%%%%%%%%%%%%%%%%%%%%%%%%%%%%%%%%%%%%%%%%%%%%%%%%%%%%
\usepackage{color}
\definecolor{Valentia}{RGB}{233,78,82}
\definecolor{Titleblue}{RGB}{114,146,162}

%%%%%%%%%%%%%%%%%%%%%%%%%%%%%%%%%%%%%%%%%%%%%%%%%%%%%%%%%%%%%%%%
% LAYOUT AND FORMATTING PACKAGES
%%%%%%%%%%%%%%%%%%%%%%%%%%%%%%%%%%%%%%%%%%%%%%%%%%%%%%%%%%%%%%%%
\usepackage{mdframed}
\usepackage{multirow}
\usepackage{multicol}
\usepackage{tikz}
\usepackage{graphicx}
\usepackage[absolute]{textpos}
\usepackage{colortbl}
\usepackage{array}
\usepackage{geometry}
\usepackage{scrextend}

%%%%%%%%%%%%%%%%%%%%%%%%%%%%%%%%%%%%%%%%%%%%%%%%%%%%%%%%%%%%%%%%
% HEADER AND FOOTER SETTINGS
%%%%%%%%%%%%%%%%%%%%%%%%%%%%%%%%%%%%%%%%%%%%%%%%%%%%%%%%%%%%%%%%
\usepackage{fancyhdr}
\pagestyle{fancy}
\fancyhf{}
\fancyheadoffset{0.005\textwidth}
\fancyhead[LE]{\thepage}
\fancyhead[RE]{\nouppercase\leftmark}
\fancyhead[RO]{\thepage}
\fancyhead[LO]{\nouppercase\rightmark}

%%%%%%%%%%%%%%%%%%%%%%%%%%%%%%%%%%%%%%%%%%%%%%%%%%%%%%%%%%%%%%%%
% HYPERLINKS AND CHAPTER STYLES
%%%%%%%%%%%%%%%%%%%%%%%%%%%%%%%%%%%%%%%%%%%%%%%%%%%%%%%%%%%%%%%%
\usepackage{hyperref}
\usepackage[Sonny]{fncychap}
\usepackage[colorlinks=true, linkcolor=blue, citecolor=red, urlcolor=blue]{hyperref}

%%%%%%%%%%%%%%%%%%%%%%%%%%%%%%%%%%%%%%%%%%%%%%%%%%%%%%%%%%%%%%%%
% BEGINNING OF THE DOCUMENT
%%%%%%%%%%%%%%%%%%%%%%%%%%%%%%%%%%%%%%%%%%%%%%%%%%%%%%%%%%%%%%%%
\begin{document}

%%%%%%%%%%%%%%%%%%%%%%%%%%%%%%%%%%%%%%%%%%%%%%%%%%%%%%%%%%%%%%%%
% TITLE PAGE
%%%%%%%%%%%%%%%%%%%%%%%%%%%%%%%%%%%%%%%%%%%%%%%%%%%%%%%%%%%%%%%%
\begin{titlepage}
\newgeometry{left=2.5cm, bottom=3cm, top=2cm, right=2.5cm}

\tikz[remember picture,overlay] \node[opacity=0.03,inner sep=0pt] at (73.6mm, -124.25mm){\includegraphics{characters.png}};

\centering
\color{black}
\fontsize{24}{13}\selectfont
\textbf{DOKUMENT PROJEKTU} \\[2mm]
\normalsize
\color{black}
\bigskip
\textbf{Wersja dokumentu: 1.1}\\[1mm]
\bigskip
\textbf{Data utworzenia: 19.03.2025}\\[1mm]
\bigskip
\textbf{Data ostatniej aktualizacji: 25.03.2025}

% Title of the project
\color{black}
\vspace{2cm}
{\fontsize{28}{32} \selectfont \textbf{Gra internetowa}}\\ 
\vspace{0.3cm} 
{\fontsize{45}{32} \selectfont \textbf{Codenames}} 

% Subtitle of the project
\vspace{2cm}
\fontsize{15}{18}\selectfont
\color{black}
\textbf{Organizacja i infrastruktura projektu\\}
\bigskip
\vspace{5cm}

% Information about authors
\normalsize
\bigskip
\fontsize{12}{12}\selectfont
\vspace{1.5mm}
\raggedright
\begin{tabular}{ll}
\textbf{Redaktor:} & Zuzanna Nowak, s193165 \\[6mm]
\textbf{Współautorzy:} & Agata Domasik, 193577 \\[2mm]
& Jakub Walasik, s193650 \\[2mm]
& Adam Chabraszewski, s193373 \\[6mm]
\textbf{Liczba stron:} & 24 \\[2mm]
\end{tabular}

% Logo
\vspace{\fill}
\begin{center}
    \includegraphics[scale=0.3]{logo.png} 
\end{center}
\vspace{-15mm}
\end{titlepage}

%%%%%%%%%%%%%%%%%%%%%%%%%%%%%%%%%%%%%%%%%%%%%%%%%%%%%%%%%%%%%%%%
% GEOMETRY SETTINGS FOR THE MAIN BODY OF THE DOCUMENT
%%%%%%%%%%%%%%%%%%%%%%%%%%%%%%%%%%%%%%%%%%%%%%%%%%%%%%%%%%%%%%%%
\newgeometry{top=2cm, bottom=2.5cm, left=2cm, right=2cm}

%%%%%%%%%%%%%%%%%%%%%%%%%%%%%%%%%%%%%%%%%%%%%%%%%%%%%%%%%%%%%%%%
% CONTENT OF THE DOCUMENT
%%%%%%%%%%%%%%%%%%%%%%%%%%%%%%%%%%%%%%%%%%%%%%%%%%%%%%%%%%%%%%%%

%%%%%%%%%%%%%%%%%%%%%%%%%%%%%%%%%%%%%%%%%%%%%%%%%%%%%%%%%%%%%%%%
% TABLE OF CONTENTS
%%%%%%%%%%%%%%%%%%%%%%%%%%%%%%%%%%%%%%%%%%%%%%%%%%%%%%%%%%%%%%%%
\tableofcontents

%%%%%%%%%%%%%%%%%%%%%%%%%%%%%%%%%%%%%%%%%%%%%%%%%%%%%%%%%%%%%%%%
% Chapters
%%%%%%%%%%%%%%%%%%%%%%%%%%%%%%%%%%%%%%%%%%%%%%%%%%%%%%%%%%%%%%%%
\chapter{Wprowadzenie - o dokumencie}
\section{Cel dokumentu}
Celem dokumentu jest uporządkowanie podstawowych informacji o infrastrukturze projektu, organizacji pracy zespołu, rodzajach komunikacji czy sposobie wymiany dokumentów i kodu między członkami zespołu. 

\section{Odbiorcy}

\begin{itemize}
    \item Dr inż. Jakub Miler - prowadzący przedmiot \textit{Realizacja projektu informatycznego},
    \item Dr inż. Katarzyna Łukasiewicz - prowadzący zajęcia projektowe,
    \item Katedra Inżynierii Oprogramowania, \\[2mm] 
Wydział Elektroniki, Telekomunikacji i Informatyki, \\[2mm]  
Politechnika Gdańska,
    \item Członkowie zespołu projektowego:
    \begin{itemize}
        \item[] Zuzanna Nowak, 193165 - kierownik projektu
        \item[] Agata Domasik, 193577
        \item[] Jakub Walasik, s193650
        \item[] Adam Chabraszewski, s193373
    \end{itemize}
\end{itemize}

\section{Terminologia}
\begin{itemize}
    \item \textbf{Streaming} – transmitowanie rozgrywki na żywo za pośrednictwem platform internetowych, takich jak Twitch, YouTube czy Facebook Gaming. Umożliwia interakcję z widzami i popularyzację gry.
    \item \textbf{Turnieje online} – rywalizacja graczy w sieci w zorganizowanych zawodach, często z ustalonymi zasadami i nagrodami. Może być prowadzona na dedykowanych platformach turniejowych.
    \item \textbf{Rozproszona aplikacja} – aplikacja, której komponenty działają na różnych urządzeniach lub serwerach, komunikując się za pomocą sieci.
    \item \textbf{Sesja gry} – jednostka rozgrywki, w której gracze dołączają do wspólnej gry według określonych zasad.
    \item \textbf{Kapitan drużyny} – gracz odpowiedzialny za podawanie wskazówek w grze „Tajniacy”.
    \item \textbf{Interfejs użytkownika (UI)} – część aplikacji, z którą użytkownicy wchodzą w interakcję, np. menu gry, przyciski czy plansza gry.
    \item \textbf{Backend} – część systemu odpowiedzialna za przetwarzanie danych i logikę aplikacji, działająca po stronie serwera.
    \item \textbf{Frontend} – część systemu odpowiedzialna za interakcję użytkownika, działająca po stronie przeglądarki lub aplikacji klienckiej.
    \item \textbf{WebSocket} – technologia umożliwiająca dwukierunkową komunikację w czasie rzeczywistym pomiędzy klientem a serwerem.
    \item \textbf{Repozytorium Git} – system kontroli wersji używany do przechowywania i zarządzania kodem źródłowym projektu.
\end{itemize}

\chapter{Opis projektu}

\section{Nazwa projektu}
Rozproszona aplikacja do rozgrywek w grę imprezową `Tajniacy`.

\section{Adresowany problem}
Projekt odpowiada na potrzebę stworzenia cyfrowej wersji popularnej gry planszowej "Tajniacy", umożliwiającej rozgrywkę bez posiadania fizycznej wersji gry. Obecne na rynku implementacje nie oddają w pełni mechaniki oryginalnej gry, co utrudnia rozgrywkę i pogarsza wrażenia z gry.

\section{Obszar zastosowania}
Gra znajduje zastosowanie w obszarach:
\begin{itemize}
    \item Rozrywka i gry towarzyskie online, 
    \item Trening logicznego myślenia, 
    \item Narzędzia do integracji zespołów,
    \item Streaming i turnieje online
\end{itemize}

\section{Rynek}
Docelowym rynkiem dla tego produktu są różnorodne grupy odbiorców:
\begin{itemize}
    \item Gracze gier planszowych, 
    \item Fani gier słownych i łamigłówek, 
    \item Osoby poszukujące rozgrywki w formie cyfrowej, 
    \item Zespoły pracownicze szukające narzędzi do integracji zespołu,
    \item Edukatorzy wykorzystujący gry jako narzędzia dydaktyczne,
    \item Rodziny, szukające rozrywki dla różnych grup wiekowych
\end{itemize}

\section{Interesariusze}
Interesariuszami projektu są:
\begin{itemize}
    \item \textbf{Deweloperzy} odpowiedzialni za implementację, 
    \item \textbf{Testerzy} weryfikujący funkcjonalności 
    \item \textbf{Gracze}
\end{itemize}

\section{Użytkownicy i ich potrzeby}
Użytkownikami końcowymi produktu będą:
\begin{itemize}
    \item gracze indywidualni - wymagają możliwości dołączania do publicznych rozgrywek, 
    \item grupy znajomych - wymagają możliwości utworzenia rozgrywki prywatnej oraz zaproszania do niej graczy
\end{itemize}
Kluczowymi potrzebami wyżej wymienionych użytkowników są: prosty, intuicyjny interfejs oraz możliwość prowadzenia rozgrywki zdalnie.

\section{Cel i zakres projektu}
Celem projektu jest stworzenie rozproszonej wersji gry „Tajniacy”, umożliwiającej graczom rywalizację w czasie rzeczywistym z różnych lokalizacji. Gra będzie dostępna online, co pozwoli na granie w różnych środowiskach i na różnych urządzeniach. Rozgrywka opiera się na współpracy drużynowej oraz analizie słów. Drużyny będą próbować odgadnąć słowa powiązane z podpowiedziami kapitana, zgodnie z zasadami oryginalnej gry. System zostanie zaprojektowany tak, by rozwijać strategiczne myślenie, umiejętność kojarzenia słów oraz pracę zespołową. 

\section{Ograniczenia}
\begin{itemize}
    \item \textbf{Czas realizacji}: Projekt ma ograniczony czas realizacji – dwa semestry. 
    \item \textbf{Zasoby ludzkie}: Ograniczona liczba programistów.
    \item \textbf{Budżet}: Ograniczone fundusze na hosting i narzędzia. 
\end{itemize}

\section{Inne współpracujące systemy}
Produkt będzie współpracował z serwerami obsługującymi rozgrywkę online, bazami danych przechowującymi informacje o użytkownikach i rozgrywkach oraz systemami autoryzacji zapewniającymi bezpieczny dostęp.

\section{Harmonogram}
\subsection{Termin}
Projekt zostanie zrealizowany do 15 czerwca 2025 roku.

\subsection{Główne etapy projektu}
Główne etapy realizacji projektu obejmują:
\begin{itemize}
    \item Analiza i planowanie
    \begin{itemize}
        \item Analiza wymagań - zbieranie wymagań funkcjonalnych i niefunkcjonalnych, opracowywanie scenariuszy użytkowników.
        \item Planowanie techniczne - podjęcie decyzji na temat technologii i narzędzi. Projektowanie architektury aplikacji oraz przydzielenie zadań i ról w zespole.
    \end{itemize}
    \item Faza prototypowania
    \begin{itemize}
        \item Prototyp Backend - konfiguracja podstawowego serwera Spring Boot. Implementacja protokołu WebSockets do komunikacji w czasie rzeczywistym.
        \item Prototyp Frontend - opracowanie ekranów gry, stworzenie prostego interfejsu dołączania do sesji gry.
    \end{itemize}
    \item Etap realizacji
    \begin{itemize}
        \item Podstawowa logika gry - implementacja mechaniki rozgrywki wieloosobowej, możliwości stworzenia sesji gry i logiki komunikacji między graczami.
        \item System sesyjny - implementacja możliwości tworzenia i dołączania do sesji gry
    \end{itemize}
    \item Testowanie i weryfikacja
    \begin{itemize}
        \item Testy funkcjonalne - testy funkcjonalności oraz integracyjne
    \end{itemize}
    \item Opracowanie dokumentacji
\end{itemize}

\clearpage

\chapter{Interesariusze i użytkownicy}

\section{Interesariusze}
Interesariuszami projektu są:
\begin{itemize}
    \item \textbf{Deweloperzy} - odpowiedzialni za implementację systemu,
    \item \textbf{Testerzy} - odpowiedzialni za weryfikację funkcjonalności,
    \item \textbf{Gracze} - użytkownicy oprogramowania
\end{itemize}

\section{Użytkownicy końcowi}
Użytkownikami końcowymi produktu będą:
\begin{itemize}
    \item \textbf{Gracze indywidualni} - osoby korzystające z systemu do uczestnictwa w publicznych rozgrywkach,
    \item \textbf{Grupy znajomych} - zespoły osób pragnących prowadzić prywatne rozgrywki w zamkniętym gronie
\end{itemize}

\section{Klasyfikacja i krótki opis interesariuszy}
\subsection{Interesariusze wewnętrzni}
\begin{itemize}
    \item \textbf{Deweloperzy} - zespół programistów zaangażowanych w tworzenie systemu,
    \item \textbf{Testerzy} - odpowiedzialni za weryfikację funkcjonalności
\end{itemize}

\subsection{Interesariusze zewnętrzni}
\begin{itemize}
    \item \textbf{Gracze indywidualni} - użytkownicy poszukujący możliwości uczestnictwa w publicznych rozgrywkach z innymi graczami.
    \item \textbf{Grupy znajomych} - zespoły użytkowników pragnących organizować prywatne rozgrywki.
    \item \textbf{Prawo UE i prawo narodowe} - zbiór regulacji prawnych, które określają zasady działania systemu i jego zgodność z przepisami.
\end{itemize}

\chapter{Zespół}
\section{Skład zespołu}
\begin{itemize}
        \item Zuzanna Nowak, 193165 - kierownik projektu
        \item Agata Domasik, 193577
        \item Jakub Walasik, s193650
        \item Adam Chabraszewski, s193373
\end{itemize}

\section{Kompetencje i umiejętności członków zespołu}
\begin{itemize}
  \item \textbf{Zuzanna Nowak}:
    \begin{itemize}
        \item \textit{Umiejętności techniczne:} 
        \begin{itemize}
            \item \textbf{Języki programowania:} Java, C++, C, C\#, JavaScript, TypeScript, PHP, Python, Bash, Język asemblera
            \item \textbf{Frameworki:} Spring Framework, Spring Boot, React.js, AngularJS, Windows Presentation Foundation (WPF), Unity, Vue.js
            \item \textbf{Bazy danych:} MySQL, MongoDB, Microsoft SQL Server, SQL, ELK Stack
            \item \textbf{Front-end:} HTML5, CSS
            \item \textbf{DevOps:} Ansible, Unix, Microsoft Azure, Git, Docker, Docker-Compose, Kubernetes
            \item \textbf{Monitoring i serwery:} Grafana, ELK stack monitoring, Web Servers (Apache, Tomcat)
            \item \textbf{Inne:} Algorytmy, Struktury danych
        \end{itemize}
        \item \textit{Umiejętności dodatkowe:} Adobe Photoshop, Adobe Premiere Pro, Figma
        \item \textit{Języki obce:} Angielski (Certificate in Advanced English CAE)
        \item \textit{Edukacja:} Informatyka (studia inżynierskie – w trakcie)
        \item \textit{Doświadczenie zawodowe:} DevOps Engineer w firmie Vention
    \end{itemize}
    
    \item \textbf{Agata Domasik}:
    \begin{itemize}
        \item \textit{Umiejętności techniczne:} 
        \begin{itemize}
            \item \textbf{Języki programowania:} Java, C++, C, C\#, JavaScript, TypeScript, PHP, Python, Bash, Język asemblera
            \item \textbf{Frameworki:} Spring Framework, Spring Boot, React.js, AngularJS, Windows Presentation Foundation (WPF), Unity, Vue.js
            \item \textbf{Bazy danych:} MySQL, MongoDB, Microsoft SQL Server, SQL
            \item \textbf{Front-end:} HTML5, CSS
            \item \textbf{DevOps:} Git
            \item \textbf{Inne:} Algorytmy, Struktury danych
        \end{itemize}
        \item \textit{Umiejętności dodatkowe:} Figma
        \item \textit{Języki obce:} Angielski (Certificate in Advanced English CAE), Niemiecki (poziom A1+)
        \item \textit{Edukacja:} Informatyka (studia inżynierskie – w trakcie)
    \end{itemize}
    
    \item \textbf{Jakub Walasik}:
    \begin{itemize}
        \item \textit{Umiejętności techniczne:} 
        \begin{itemize}
            \item \textbf{Języki programowania:} Java, C++, C, C\#, JavaScript, TypeScript, PHP, Python, Język asemblera
            \item \textbf{Frameworki:} Spring Framework, Spring Boot, React.js, AngularJS
            \item \textbf{Bazy danych:} MySQL, PostgreSQL, MongoDB, Microsoft SQL Server, SQL
            \item \textbf{Front-end:} HTML5, CSS, SCSS
            \item \textbf{DevOps:} Git
            \item \textbf{Inne:} Algorytmy, Struktury danych
        \end{itemize}
        \item \textit{Języki obce:} Angielski (poziom C1)
        \item \textit{Edukacja:} Informatyka (studia inżynierskie – w trakcie)
    \end{itemize}
    
    \item \textbf{Adam Chabraszewski}:
    \begin{itemize}
        \item \textit{Umiejętności techniczne:} 
        \begin{itemize}
            \item \textbf{Języki programowania:} Java, C++, C, C\#, JavaScript, TypeScript, PHP, Python, Bash, Język asemblera
            \item \textbf{Frameworki:} Spring Framework, Spring Boot, React.js, AngularJS, Unity, Vue.js
            \item \textbf{Bazy danych:} MySQL, MongoDB, Microsoft SQL Server, SQL
            \item \textbf{Front-end:} HTML5, CSS
            \item \textbf{DevOps:} Git, Unix, Docker, Docker-Compose
            \item \textbf{Inne:} Algorytmy, Struktury danych
        \end{itemize}
        \item \textit{Umiejętności dodatkowe:} konfiguracja CUDA
        \item \textit{Języki obce:} Angielski (Certificate in Advanced English CAE)
        \item \textit{Edukacja:} Informatyka (studia inżynierskie – w trakcie)
        \item \textit{Doświadczenie zawodowe:} Software Developer w firmie Advanced Protection Systems Inc.
    \end{itemize}
\end{itemize}

\section{Zakresy odpowiedzialności}
\begin{center}
\begin{tabular}{|p{4.5cm}|p{8cm}|}
\hline
\textbf{Imię i naziwsko} & \textbf{Zakres odpowiedzialności} \\
\hline
Zuzanna Nowak &  Backend, frontend, DevOps, design UI/UX, kierowanie zespołem, zarządzanie repozytorium\\
\hline
Agata Domasik &  Backend, frontend, design UI/UX\\
\hline
Jakub Walasik &  Backend, frontend, DevOps, testowanie\\
\hline
Adam Chabraszewski & Backend, frontend\\
\hline
\end{tabular}
\end{center}

\section{Model pracy}
Zespół pracuje w modelu zdalnym:
\begin{itemize}
    \item Spotkania zespołowe odbywają się przez platformę Discord na dedykowanym serwerze projektu
    \item Regularne spotkania projektowe mają miejsce co tydzień we wtorek o godzinie 19:00
    \item Do zarządzania zadaniami zespół wykorzystuje GitHub Projects z tablicą Kanban
    \item Postępy prac są na bieżąco aktualizowane w systemie kontroli wersji GitHub
    \item Członkowie zespołu wykonują swoje zadania asynchronicznie, utrzymując stały kontakt przez Discord oraz Messenger
\end{itemize}

\section{Dane kontaktowe}
\begin{itemize}
    \item Oficjalny e-mail kontaktowy zespołu: 
    \href{mailto:codenames.contact@gmail.com}{codenames.contact@gmail.com}
    \item Dane kontaktowe członków zespołu:
    \begin{itemize}
        \item \textbf{Zuzanna Nowak}:
        \begin{itemize}
            \item E-mail: \href{mailto:zuzanna.nowak.kontakt@gmail.com}{zuzanna.nowak.kontakt@gmail.com} 
            \item LinkedIn: \url{https://www.linkedin.com/in/zuzanna-nowak-aa071125b/} 
            \item GitHub: \url{https://github.com/zuzqqa}
            \item Facebook: \url{https://www.facebook.com/zuzia.nowak.50951/}
        \end{itemize}
        \item \textbf{Agata Domasik}:
        \begin{itemize}
            \item E-mail: \href{mailto:agata.domasik@gmail.com}{agata.domasik@gmail.com} 
            \item LinkedIn: \url{https://www.linkedin.com/in/agata-domasik-50277b2bb/}
            \item GitHub: \url{https://github.com/agatadomasik}
            \item Facebook: \url{https://www.facebook.com/agata.domasik}
        \end{itemize}
        \item \textbf{Jakub Walasik}:
        \begin{itemize}
            \item E-mail: \href{mailto:walasikjakub@gmail.com}{walasikjakub@gmail.com}  
            \item LinkedIn: \url{https://www.linkedin.com/in/jakub-walasik-9ba6b9348/}
            \item GitHub: \url{https://github.com/jwalasik3}
            \item Facebook: \url{https://www.facebook.com/jakub.walasik.5}
        \end{itemize}
        \item \textbf{Adam Chabraszewski}:
        \begin{itemize}
            \item E-mail: \href{mailto:adam.chabraszewski@gmail.com}{adam.chabraszewski@gmail.com}   
            \item LinkedIn: \url{https://www.linkedin.com/in/adam-chabraszewski-574a1530a/}
            \item GitHub: \url{https://github.com/achabrasz}
            \item Facebook: \url{https://www.facebook.com/adam.chabraszewski.92}
        \end{itemize}
    \end{itemize}
\end{itemize}

\chapter{Komunikacja w zespole i z interesariuszami}
\section{Harmonogram spotkań}
Zespół spotyka się regularnie co tydzień we wtorek o godzinie 19:00 na serwerze Discord.  

\section{Miejsca spotkań}
Spotkania odbywają się na serwerze Discord.

\section{Narzędzia komunikacji}
Komunikacja zespołu odbywa się głównie za pośrednictwem:  
\begin{itemize}
    \item Serwera Discord – wykorzystywanego zarówno do spotkań, jak i bieżącej komunikacji tekstowej i głosowej.  
    \item Grupy na Messengerze – używanej do szybkich ustaleń i organizacji.  
\end{itemize}  

\section{Komunikacja z interesariuszami zewnętrznymi}
Kontakt z interesariuszami zewnętrznymi odbywa się poprzez spotkania na żywo.

\section{Współpraca z opiekunem projektu}
Spotkania z opiekunem projektu odbywają się nieregularnie, w zależności od potrzeby. Podczas tych spotkań zespół zdaje raport z postępów prac oraz uzyskuje odpowiedzi na ewentualne pytania.  

\chapter{Współdzielenie dokumentów i kodu}
\section{Metody wymiany dokumentów i kodu}
Kod i dokumentacja projektu są współdzielone poprzez repozytorium GitHub. Każde zadanie jest rejestrowane jako \textbf{Issue}, po czym tworzona jest dla niego osobna gałąź (\textit{branch}). Po zakończeniu pracy nad zadaniem składany jest \textbf{Pull Request} do głównej gałęzi (\textit{main}).  
\section{Repozytorium projektu}
Projekt jest przechowywany w repozytorium na GitHubie, które zawiera zarówno kod źródłowy, jak i dokumentację.  

\subsection{Administracja repozytorium}
Administracją repozytorium zajmuje się \textbf{kierownik projektu}, który odpowiada za nadzór nad kodem, zatwierdzanie Pull Requestów oraz kontrolę dostępu. Gałąź \textit{main} jest zablokowana i nie można do niej bezpośrednio dodawać zmian – każda zmiana musi przejść proces Pull Requesta.  

\section{Zarządzanie dokumentacją}
Dokumentacja projektu jest przechowywana w repozytorium w dedykowanym folderze \texttt{docs}.  

\subsection{Osoba odpowiedzialna}
Dokumentacją zajmują się wszyscy członkowie zespołu, jednak nadzór nad jej poprawnością sprawuje kierownik projektu.  

\subsection{Konwencje nazewnictwa}  
Pliki dokumentacji są nazywane zgodnie z następującymi zasadami:  
\begin{itemize}  
    \item Nazwa pliku powinna odpowiadać tytułowi dokumentu, być w języku angielskim.  
    \item Stosujemy format \texttt{snake\_case} dla nazw plików 
    (np. \texttt{project\_organization.tex}).  
\end{itemize}  

\subsection{Szablony dokumentów}
Szablony dokumentów znajdują się w repozytorium w katalogu \texttt{docs/templates}. 

\section{System wersjonowania}
Projekt wykorzystuje automatyczny system wersjonowania do zarządzania wersjami kodu.

\subsection{Wersjonowanie kodu}
Przed utworzeniem Pull Requesta należy uruchomić skrypt wersjonowania, który generuje nową wersję. Po zatwierdzeniu zmian, GitHub Actions automatycznie przypisuje nowy numer wersji do commita w gałęzi \textit{main}.  

\subsection{Wersjonowanie dokumentacji}
Dokumentacja jest wersjonowana ręcznie. Każda istotna zmiana powinna być odnotowana poprzez aktualizację numeru wersji dokumentu.

\chapter{Narzędzia}

\section{Narzędzia komunikacyjne}  
W zespole wykorzystywane są następujące narzędzia do komunikacji:  
\begin{itemize}  
    \item \textbf{Discord} – główna platforma do spotkań zespołu oraz bieżącej komunikacji głosowej i tekstowej.  
    \item \textbf{Messenger} – używany do szybkich ustaleń oraz nieformalnej komunikacji zespołowej.  
\end{itemize}  

\section{Narzędzia do zarządzania dokumentacją}  
Dokumentacja projektu jest tworzona w formacie \LaTeX{} z wykorzystaniem następujących narzędzi:  
\begin{itemize}  
    \item \textbf{GitHub} – repozytorium projektu zawiera dokumentację w katalogu \texttt{docs}.  
    \item \textbf{Overleaf} – niektórzy członkowie zespołu korzystają z niego do edycji dokumentacji w \LaTeX{}.  
\end{itemize}  

\section{Narzędzia do zarządzania projektem}  
Zarządzanie projektem odbywa się za pomocą:  
\begin{itemize}  
    \item \textbf{GitHub Projects} – tablica Kanban do śledzenia postępów prac, planowania zadań i zarządzania backlogiem.  
    \item \textbf{GitHub Issues} – wykorzystywane do zgłaszania i monitorowania zadań oraz błędów.  
\end{itemize}

\section{Narzędzia developerskie}  
Podczas pracy nad projektem wykorzystywane są następujące narzędzia programistyczne:  
\begin{itemize}  
    \item \textbf{Visual Studio Code} – główny edytor dla większości języków używanych w projekcie.  
    \item \textbf{IntelliJ IDEA} – środowisko programistyczne do pracy nad kodem w języku Java.  
    \item \textbf{Docker Desktop} – platforma do zarządzania kontenerami Docker.  
    \item \textbf{WSL (Windows Subsystem for Linux)} – umożliwia korzystanie z narzędzi linuksowych w systemie Windows.  
    \item \textbf{Git} – system kontroli wersji używany do zarządzania kodem źródłowym.  
    \item \textbf{MongoDB Compass} – narzędzie do przeglądania i zarządzania bazą danych MongoDB.  
\end{itemize}  

\section{Narzędzia testowe}  
Testowanie aplikacji odbywa się za pomocą testów jednostkowych i integracyjnych:  
\begin{itemize}  
    \item \textbf{JUnit} – framework do testów jednostkowych w języku Java.  
    \item \textbf{Testy integracyjne} – realizowane w środowisku testowym z wykorzystaniem narzędzi do symulacji warunków produkcyjnych.  
\end{itemize}  

\newpage
\thispagestyle{empty}
\null
\newpage

%%%%%%%%%%%%%%%%%%%%%%%%%%%%%%%%%%%%%%%%%%%%%%%%%%%%%%%%%%%%%%%%
% END PAGE
%%%%%%%%%%%%%%%%%%%%%%%%%%%%%%%%%%%%%%%%%%%%%%%%%%%%%%%%%%%%%%%%
\newpage
\thispagestyle{empty}
\tikz[remember picture,overlay] \node[opacity=0.03,inner sep=0pt] at (73.6mm, -124.25mm){\includegraphics{characters.png}};
\begin{center}
    \vspace*{\fill}
    \includegraphics[width=0.3\textwidth]{logo.png} 
    \vspace*{\fill}
\end{center}

%%%%%%%%%%%%%%%%%%%%%%%%%%%%%%%%%%%%%%%%%%%%%%%%%%%%%%%%%%%%%%%%
% END OF DOCUMENT
%%%%%%%%%%%%%%%%%%%%%%%%%%%%%%%%%%%%%%%%%%%%%%%%%%%%%%%%%%%%%%%%
\end{document}